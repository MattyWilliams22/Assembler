\documentclass[11pt]{article}

\usepackage{fullpage}

\begin{document}

\title{ARM Checkpoint... }
\author{Dina Duong, Maciej Rytlewski, Kritik Pant, Matty Williams}

\maketitle

\section{Group Organisation}

\subsection*{Task Allocation}

\begin{itemize}
    \item Kritik Pant: Memory initialisation, instruction processing, and data processing immediate and registers instructions.
    \item Maciej Rytlewski: Single data transfer instructions and bug fixes.
    \item Dina Duong: Branch instructions, bug fixes and report. 
    \item Matty Williams: Suggested code improvements and moved on to Part 2 (Assembler).
\end{itemize}
To coordinate our work, we made a separate branch in the Git Repository, where we experimented with the code. 
After fixing all the errors, we pushed stable working versions to the master branch. 
As the first part is sequential, the initial task was to decide how we want to implement the memory. 
After the discussion, memory initialisation was assigned to Kritik. Once the memory was set up, we all started 
implementing the instructions, dividing the workload equally between the group members. 
After completing the full functionality, Dina and Maciej focused on bug fixing while Kritik and Matty moved on to Part 2.

\subsection*{Group Evaluation}
Overall, our group has been working well together. Since Part 1 involved a sequential implementation of the emulator, 
there were limited opportunities for parallel work. However, the division of tasks among group members helped us progress efficiently.
Regular communication was maintained through discussions, ensuring everyone was aware of the progress and tasks assigned to each member.
\newline
As we progress, we anticipate the need for better task coordination and clearer communication regarding individual responsibilities. 
As the complexity of the project increases, we may need to allocate tasks based on members' expertise and consider potential 
dependencies between different components. Regular progress updates and meetings will be crucial to ensure 
everyone is on the same page and to address any challenges promptly.

\section{Implementation Strategies}

\subsection*{Emulator Structure}

We used a modular approach by splitting the code into separate files. Our file structure included:
\begin{itemize}
    \item Setup File: This file handled memory initialisation, processing instructions, and other setup-related operations. 
        We implemented memory as an array and used structs for modelling PSTATE and state of the memory and registers.
    \item Instructions File: Each instruction was implemented as a separate function within this file, allowing for easy 
        modification and addition of new instructions.
    \item Utilities File: We created a utilities file containing commonly used functions, such as 32-bit and 64-bit mode handling 
        and sign extension. These utility functions were reusable across various instructions and improved code organisation.
    \item Main File: The entry point for running the emulator.
\end{itemize}

\subsection*{Reusability for Assembler}
In our opinion, some of the utils functions could be used in the Assembler. As the Emulator is reading binary and executing it 
and Assembler is translating Assembly into binary, we do not think Emulator code can be directly reused for the Assembler.

\subsection*{Future Challenges}
\begin{itemize}
    \item Splitting Up the Work: As mentioned earlier, the project is somewhat sequential. 
    Making it hard to parallelise the work since certain tasks need to be completed before moving on to others. 
    To address this, we have been working collaboratively and there has been a lot of communication, making sure we let each other know
    what we have been doing and on what stage of implementation we are.
    \item Reducing and Cleaning Up the Code: As the code length increases, it has been hard to manage and maintain it. To mitigate this,
    we have been removing duplicated and unused code. We have been having code reviews, during which we indentified areas of improvement.
\end{itemize}

\end{document}
